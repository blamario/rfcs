\haddockmoduleheading{System.IO.Error}
\label{module:System.IO.Error}
\haddockbeginheader
{\haddockverb\begin{verbatim}
module System.IO.Error (
    IOError,  userError,  mkIOError,  annotateIOError,  isAlreadyExistsError, 
    isDoesNotExistError,  isAlreadyInUseError,  isFullError,  isEOFError, 
    isIllegalOperation,  isPermissionError,  isUserError,  ioeGetErrorString, 
    ioeGetHandle,  ioeGetFileName,  IOErrorType,  alreadyExistsErrorType, 
    doesNotExistErrorType,  alreadyInUseErrorType,  fullErrorType, 
    eofErrorType,  illegalOperationErrorType,  permissionErrorType, 
    userErrorType,  ioError,  catch,  try
  ) where\end{verbatim}}
\haddockendheader

\section{I/O errors
}
\begin{haddockdesc}
\item[\begin{tabular}{@{}l}
type\ IOError\ =\ IOError
\end{tabular}]\haddockbegindoc
Errors of type \haddockid{IOError} are used by the \haddockid{IO} monad.  This is an
 abstract type; the module \haddocktt{System.IO.Error} provides functions to
 interrogate and construct values of type \haddockid{IOError}.
\par

\end{haddockdesc}
\begin{haddockdesc}
\item[\begin{tabular}{@{}l}
userError\ ::\ String\ ->\ IOError
\end{tabular}]\haddockbegindoc
Construct an \haddockid{IOError} value with a string describing the error.
 The \haddockid{fail} method of the \haddockid{IO} instance of the \haddockid{Monad} class raises a
 \haddockid{userError}, thus:
\par
\begin{quote}
{\haddockverb\begin{verbatim}
 instance Monad IO where 
   ...
   fail s = ioError (userError s)
\end{verbatim}}
\end{quote}

\end{haddockdesc}
\begin{haddockdesc}
\item[\begin{tabular}{@{}l}
mkIOError\ ::\ IOErrorType\\\ \ \ \ \ \ \ \ \ \ \ \ \ ->\ String\ ->\ Maybe\ Handle\ ->\ Maybe\ FilePath\ ->\ IOError
\end{tabular}]\haddockbegindoc
Construct an \haddockid{IOError} of the given type where the second argument
 describes the error location and the third and fourth argument
 contain the file handle and file path of the file involved in the
 error if applicable.
\par

\end{haddockdesc}
\begin{haddockdesc}
\item[\begin{tabular}{@{}l}
annotateIOError\ ::\ IOError\\\ \ \ \ \ \ \ \ \ \ \ \ \ \ \ \ \ \ \ ->\ String\ ->\ Maybe\ Handle\ ->\ Maybe\ FilePath\ ->\ IOError
\end{tabular}]\haddockbegindoc
Adds a location description and maybe a file path and file handle
 to an \haddockid{IOError}.  If any of the file handle or file path is not given
 the corresponding value in the \haddockid{IOError} remains unaltered.
\par

\end{haddockdesc}
\subsection{Classifying I/O errors
}
\begin{haddockdesc}
\item[\begin{tabular}{@{}l}
isAlreadyExistsError\ ::\ IOError\ ->\ Bool
\end{tabular}]\haddockbegindoc
An error indicating that an \haddockid{IO} operation failed because
 one of its arguments already exists.
\par

\end{haddockdesc}
\begin{haddockdesc}
\item[\begin{tabular}{@{}l}
isDoesNotExistError\ ::\ IOError\ ->\ Bool
\end{tabular}]\haddockbegindoc
An error indicating that an \haddockid{IO} operation failed because
 one of its arguments does not exist.
\par

\end{haddockdesc}
\begin{haddockdesc}
\item[\begin{tabular}{@{}l}
isAlreadyInUseError\ ::\ IOError\ ->\ Bool
\end{tabular}]\haddockbegindoc
An error indicating that an \haddockid{IO} operation failed because
 one of its arguments is a single-use resource, which is already
 being used (for example, opening the same file twice for writing
 might give this error).
\par

\end{haddockdesc}
\begin{haddockdesc}
\item[\begin{tabular}{@{}l}
isFullError\ ::\ IOError\ ->\ Bool
\end{tabular}]\haddockbegindoc
An error indicating that an \haddockid{IO} operation failed because
 the device is full.
\par

\end{haddockdesc}
\begin{haddockdesc}
\item[\begin{tabular}{@{}l}
isEOFError\ ::\ IOError\ ->\ Bool
\end{tabular}]\haddockbegindoc
An error indicating that an \haddockid{IO} operation failed because
 the end of file has been reached.
\par

\end{haddockdesc}
\begin{haddockdesc}
\item[\begin{tabular}{@{}l}
isIllegalOperation\ ::\ IOError\ ->\ Bool
\end{tabular}]\haddockbegindoc
An error indicating that an \haddockid{IO} operation failed because
 the operation was not possible.
 Any computation which returns an \haddockid{IO} result may fail with
 \haddockid{isIllegalOperation}.  In some cases, an implementation will not be
 able to distinguish between the possible error causes.  In this case
 it should fail with \haddockid{isIllegalOperation}.
\par

\end{haddockdesc}
\begin{haddockdesc}
\item[\begin{tabular}{@{}l}
isPermissionError\ ::\ IOError\ ->\ Bool
\end{tabular}]\haddockbegindoc
An error indicating that an \haddockid{IO} operation failed because
 the user does not have sufficient operating system privilege
 to perform that operation.
\par

\end{haddockdesc}
\begin{haddockdesc}
\item[\begin{tabular}{@{}l}
isUserError\ ::\ IOError\ ->\ Bool
\end{tabular}]\haddockbegindoc
A programmer-defined error value constructed using \haddockid{userError}.
\par

\end{haddockdesc}
\subsection{Attributes of I/O errors
}
\begin{haddockdesc}
\item[
ioeGetErrorString\ ::\ IOError\ ->\ String
]
\item[
ioeGetHandle\ ::\ IOError\ ->\ Maybe\ Handle
]
\item[
ioeGetFileName\ ::\ IOError\ ->\ Maybe\ FilePath
]
\end{haddockdesc}
\section{Types of I/O error
}
\begin{haddockdesc}
\item[\begin{tabular}{@{}l}
data\ IOErrorType
\end{tabular}]\haddockbegindoc
An abstract type that contains a value for each variant of \haddockid{IOError}.
\par

\end{haddockdesc}
\begin{haddockdesc}
\item[\begin{tabular}{@{}l}
instance\ Eq\ IOErrorType\\instance\ Show\ IOErrorType
\end{tabular}]
\end{haddockdesc}
\begin{haddockdesc}
\item[\begin{tabular}{@{}l}
alreadyExistsErrorType\ ::\ IOErrorType
\end{tabular}]\haddockbegindoc
I/O error where the operation failed because one of its arguments
 already exists.
\par

\end{haddockdesc}
\begin{haddockdesc}
\item[\begin{tabular}{@{}l}
doesNotExistErrorType\ ::\ IOErrorType
\end{tabular}]\haddockbegindoc
I/O error where the operation failed because one of its arguments
 does not exist.
\par

\end{haddockdesc}
\begin{haddockdesc}
\item[\begin{tabular}{@{}l}
alreadyInUseErrorType\ ::\ IOErrorType
\end{tabular}]\haddockbegindoc
I/O error where the operation failed because one of its arguments
 is a single-use resource, which is already being used.
\par

\end{haddockdesc}
\begin{haddockdesc}
\item[\begin{tabular}{@{}l}
fullErrorType\ ::\ IOErrorType
\end{tabular}]\haddockbegindoc
I/O error where the operation failed because the device is full.
\par

\end{haddockdesc}
\begin{haddockdesc}
\item[\begin{tabular}{@{}l}
eofErrorType\ ::\ IOErrorType
\end{tabular}]\haddockbegindoc
I/O error where the operation failed because the end of file has
 been reached.
\par

\end{haddockdesc}
\begin{haddockdesc}
\item[\begin{tabular}{@{}l}
illegalOperationErrorType\ ::\ IOErrorType
\end{tabular}]\haddockbegindoc
I/O error where the operation is not possible.
\par

\end{haddockdesc}
\begin{haddockdesc}
\item[\begin{tabular}{@{}l}
permissionErrorType\ ::\ IOErrorType
\end{tabular}]\haddockbegindoc
I/O error where the operation failed because the user does not
 have sufficient operating system privilege to perform that operation.
\par

\end{haddockdesc}
\begin{haddockdesc}
\item[\begin{tabular}{@{}l}
userErrorType\ ::\ IOErrorType
\end{tabular}]\haddockbegindoc
I/O error that is programmer-defined.
\par

\end{haddockdesc}
\section{Throwing and catching I/O errors
}
\begin{haddockdesc}
\item[\begin{tabular}{@{}l}
ioError\ ::\ IOError\ ->\ IO\ a
\end{tabular}]\haddockbegindoc
Raise an \haddockid{IOError} in the \haddockid{IO} monad.
\par

\end{haddockdesc}
\begin{haddockdesc}
\item[\begin{tabular}{@{}l}
catch\ ::\ IO\ a\ ->\ (IOError\ ->\ IO\ a)\ ->\ IO\ a
\end{tabular}]\haddockbegindoc
The \haddockid{catch} function establishes a handler that receives any \haddockid{IOError}
 raised in the action protected by \haddockid{catch}.  An \haddockid{IOError} is caught by
 the most recent handler established by \haddockid{catch}.  These handlers are
 not selective: all \haddockid{IOError}s are caught.  Exception propagation
 must be explicitly provided in a handler by re-raising any unwanted
 exceptions.  For example, in
\par
\begin{quote}
{\haddockverb\begin{verbatim}
 f = catch g (\e -> if IO.isEOFError e then return [] else ioError e)
\end{verbatim}}
\end{quote}
the function \haddocktt{f} returns \haddocktt{{\char 91}{\char 93}} when an end-of-file exception
 (cf. \haddockid{isEOFError}) occurs in \haddocktt{g}; otherwise, the
 exception is propagated to the next outer handler.
\par
When an exception propagates outside the main program, the Haskell
 system prints the associated \haddockid{IOError} value and exits the program.
\par

\end{haddockdesc}
\begin{haddockdesc}
\item[\begin{tabular}{@{}l}
try\ ::\ IO\ a\ ->\ IO\ (Either\ IOError\ a)
\end{tabular}]\haddockbegindoc
The construct \haddockid{try} \haddocktt{comp} exposes IO errors which occur within a
 computation, and which are not fully handled.
\par

\end{haddockdesc}