\haddockmoduleheading{Data.Maybe}
\label{module:Data.Maybe}
\haddockbeginheader
{\haddockverb\begin{verbatim}
module Data.Maybe (
    Maybe(Nothing, Just),  maybe,  isJust,  isNothing,  fromJust,  fromMaybe, 
    listToMaybe,  maybeToList,  catMaybes,  mapMaybe
  ) where\end{verbatim}}
\haddockendheader

\section{The \haddocktt{Maybe} type and operations
}
\begin{haddockdesc}
\item[\begin{tabular}{@{}l}
data\ Maybe\ a
\end{tabular}]\haddockbegindoc
\haddockbeginconstrs
\haddockdecltt{=} & \haddockdecltt{Nothing} & \\
\haddockdecltt{|} & \haddockdecltt{Just a} & \\
\end{tabulary}\par
The \haddockid{Maybe} type encapsulates an optional value.  A value of type
 \haddocktt{Maybe\ a} either contains a value of type \haddocktt{a} (represented as \haddocktt{Just\ a}), 
 or it is empty (represented as \haddockid{Nothing}).  Using \haddockid{Maybe} is a good way to 
 deal with errors or exceptional cases without resorting to drastic
 measures such as \haddockid{error}.
\par
The \haddockid{Maybe} type is also a monad.  It is a simple kind of error
 monad, where all errors are represented by \haddockid{Nothing}.  A richer
 error monad can be built using the \haddocktt{Data.Either.Either} type.
\par

\end{haddockdesc}
\begin{haddockdesc}
\item[\begin{tabular}{@{}l}
instance\ Functor\ Maybe\\instance\ Applicative\ Maybe\\instance\ Alternative\ Maybe\\instance\ Monad\ Maybe\\instance\ MonadPlus\ Maybe\\instance\ Eq\ a\ =>\ Eq\ (Maybe\ a)\\instance\ Ord\ a\ =>\ Ord\ (Maybe\ a)\\instance\ Read\ a\ =>\ Read\ (Maybe\ a)\\instance\ Show\ a\ =>\ Show\ (Maybe\ a)
\end{tabular}]
\end{haddockdesc}
\begin{haddockdesc}
\item[\begin{tabular}{@{}l}
maybe\ ::\ b\ ->\ (a\ ->\ b)\ ->\ Maybe\ a\ ->\ b
\end{tabular}]\haddockbegindoc
The \haddockid{maybe} function takes a default value, a function, and a \haddockid{Maybe}
 value.  If the \haddockid{Maybe} value is \haddockid{Nothing}, the function returns the
 default value.  Otherwise, it applies the function to the value inside
 the \haddockid{Just} and returns the result.
\par

\end{haddockdesc}
\begin{haddockdesc}
\item[\begin{tabular}{@{}l}
isJust\ ::\ Maybe\ a\ ->\ Bool
\end{tabular}]\haddockbegindoc
The \haddockid{isJust} function returns \haddockid{True} iff its argument is of the
 form \haddocktt{Just\ {\char '137}}.
\par

\end{haddockdesc}
\begin{haddockdesc}
\item[\begin{tabular}{@{}l}
isNothing\ ::\ Maybe\ a\ ->\ Bool
\end{tabular}]\haddockbegindoc
The \haddockid{isNothing} function returns \haddockid{True} iff its argument is \haddockid{Nothing}.
\par

\end{haddockdesc}
\begin{haddockdesc}
\item[\begin{tabular}{@{}l}
fromJust\ ::\ Maybe\ a\ ->\ a
\end{tabular}]\haddockbegindoc
The \haddockid{fromJust} function extracts the element out of a \haddockid{Just} and
 throws an error if its argument is \haddockid{Nothing}.
\par

\end{haddockdesc}
\begin{haddockdesc}
\item[\begin{tabular}{@{}l}
fromMaybe\ ::\ a\ ->\ Maybe\ a\ ->\ a
\end{tabular}]\haddockbegindoc
The \haddockid{fromMaybe} function takes a default value and and \haddockid{Maybe}
 value.  If the \haddockid{Maybe} is \haddockid{Nothing}, it returns the default values;
 otherwise, it returns the value contained in the \haddockid{Maybe}.
\par

\end{haddockdesc}
\begin{haddockdesc}
\item[\begin{tabular}{@{}l}
listToMaybe\ ::\ {\char 91}a{\char 93}\ ->\ Maybe\ a
\end{tabular}]\haddockbegindoc
The \haddockid{listToMaybe} function returns \haddockid{Nothing} on an empty list
 or \haddocktt{Just\ a} where \haddocktt{a} is the first element of the list.
\par

\end{haddockdesc}
\begin{haddockdesc}
\item[\begin{tabular}{@{}l}
maybeToList\ ::\ Maybe\ a\ ->\ {\char 91}a{\char 93}
\end{tabular}]\haddockbegindoc
The \haddockid{maybeToList} function returns an empty list when given
 \haddockid{Nothing} or a singleton list when not given \haddockid{Nothing}.
\par

\end{haddockdesc}
\begin{haddockdesc}
\item[\begin{tabular}{@{}l}
catMaybes\ ::\ {\char 91}Maybe\ a{\char 93}\ ->\ {\char 91}a{\char 93}
\end{tabular}]\haddockbegindoc
The \haddockid{catMaybes} function takes a list of \haddockid{Maybe}s and returns
 a list of all the \haddockid{Just} values. 
\par

\end{haddockdesc}
\begin{haddockdesc}
\item[\begin{tabular}{@{}l}
mapMaybe\ ::\ (a\ ->\ Maybe\ b)\ ->\ {\char 91}a{\char 93}\ ->\ {\char 91}b{\char 93}
\end{tabular}]\haddockbegindoc
The \haddockid{mapMaybe} function is a version of \haddockid{map} which can throw
 out elements.  In particular, the functional argument returns
 something of type \haddocktt{Maybe\ b}.  If this is \haddockid{Nothing}, no element
 is added on to the result list.  If it just \haddocktt{Just\ b}, then \haddocktt{b} is
 included in the result list.
\par

\end{haddockdesc}
\section{Specification
}
\begin{quote}
{\haddockverb\begin{verbatim}
 module Data.Maybe(
     Maybe(Nothing, Just),
     isJust, isNothing,
     fromJust, fromMaybe, listToMaybe, maybeToList,
     catMaybes, mapMaybe,
     maybe
   ) where
 
 maybe                  :: b -> (a -> b) -> Maybe a -> b
 maybe n _ Nothing      =  n
 maybe _ f (Just x)     =  f x
 
 isJust                 :: Maybe a -> Bool
 isJust (Just a)        =  True
 isJust Nothing         =  False
 
 isNothing              :: Maybe a -> Bool
 isNothing              =  not . isJust
 
 fromJust               :: Maybe a -> a
 fromJust (Just a)      =  a
 fromJust Nothing       =  error "Maybe.fromJust: Nothing"
 
 fromMaybe              :: a -> Maybe a -> a
 fromMaybe d Nothing    =  d
 fromMaybe d (Just a)   =  a
 
 maybeToList            :: Maybe a -> [a]
 maybeToList Nothing    =  []
 maybeToList (Just a)   =  [a]
 
 listToMaybe            :: [a] -> Maybe a
 listToMaybe []         =  Nothing
 listToMaybe (a:_)      =  Just a
  
 catMaybes              :: [Maybe a] -> [a]
 catMaybes ms           =  [ m | Just m <- ms ]
 
 mapMaybe               :: (a -> Maybe b) -> [a] -> [b]
 mapMaybe f             =  catMaybes . map f
\end{verbatim}}
\end{quote}
