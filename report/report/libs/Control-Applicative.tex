\haddockmoduleheading{Control.Applicative}
\label{module:Control.Applicative}
\haddockbeginheader
{\haddockverb\begin{verbatim}
module Control.Applicative (
    Functor(fmap),  Applicative(pure, (<*>), (*>), (<*), liftA2),
    Alternative((<|>), empty, some, many),  (<**>),  liftA,  liftA3,  optional
  ) where\end{verbatim}}
\haddockendheader

\section{Functor class
}
\begin{haddockdesc}
\item[\begin{tabular}{@{}l}
class\ Functor\ f\ where
\end{tabular}]\haddockbegindoc
The \haddockid{Functor} class is used for types that can be mapped over.
Instances of \haddockid{Functor} should satisfy the following laws:
\par
\begin{quote}
{\haddockverb\begin{verbatim}
 fmap id  ==  id
 fmap (f . g)  ==  fmap f . fmap g
\end{verbatim}}
\end{quote}
The instances of \haddockid{Functor} for lists, \haddocktt{Data.Maybe.Maybe} and \haddocktt{System.IO.IO}
satisfy these laws.
\par

\haddockpremethods{}\textbf{Methods}
\begin{haddockdesc}
\item[\begin{tabular}{@{}l}
fmap\ ::\ (a\ ->\ b)\ ->\ f\ a\ ->\ f\ b
\end{tabular}]
\end{haddockdesc}
\end{haddockdesc}
\begin{haddockdesc}
\item[\begin{tabular}{@{}l}
instance\ Functor\ {\char 91}{\char 93}\\instance\ Functor\ IO\\instance\ Functor\ Maybe\\instance\ Ix\ i\ =>\ Functor\ (Array\ i)
\end{tabular}]
\end{haddockdesc}

\section{Applicative class
}
\begin{haddockdesc}
\item[\begin{tabular}{@{}l}
class\ Functor\ f\ =>\ Applicative\ f\ where
\end{tabular}]\haddockbegindoc
A functor with application, providing operations to\par
\begin{itemize}
\item
embed pure expressions (\haddockid{pure}), and\par

\item
sequence computations and combine their results (\haddockid{<*>} and \haddockid{liftA2}).\par

\end{itemize}
A minimal complete definition must include implementations of \haddockid{pure}
 and of either \haddockid{<*>} or \haddockid{liftA2}. If it defines both, then they must behave
 the same as their default definitions:\par
\haddocktt{(<*>)\ =\ liftA2\ id}

\haddocktt{liftA2\ f\ x\ y\ =\ fmap\ f\ x\ <*>\ y}\par
Further, any definition must satisfy the following:\par
\begin{tabular}{lp{8cm}}
\emph{identity}
      & \begin{quote}
        {\haddockverb\begin{verbatim}
        pure id <*> v == v\end{verbatim}}
        \end{quote} \\

\emph{composition}
      & \begin{quote}
        {\haddockverb\begin{verbatim}
        pure (.) <*> u <*> v <*> w == u <*> (v <*> w)\end{verbatim}}
        \end{quote} \\

\emph{homomorphism}
      & \begin{quote}
        {\haddockverb\begin{verbatim}
        pure f <*> pure x == pure (f x)\end{verbatim}}
        \end{quote} \\

\emph{interchange}
      & \begin{quote}
        {\haddockverb\begin{verbatim}
        u <*> pure y == pure ($ y) <*> u\end{verbatim}}
        \end{quote}
\end{tabular}

The other methods have the following default definitions, which may
 be overridden with equivalent specialized implementations:\par
                 \begin{itemize}
                 \item
                 \begin{quote}
                 {\haddockverb\begin{verbatim}
                 u *> v = liftA2 (flip const) u v\end{verbatim}}
                 \end{quote}
                 
                 \item
                 \begin{quote}
                 {\haddockverb\begin{verbatim}
                 u <* v = liftA2 const u v\end{verbatim}}
                 \end{quote}
                 
                 \end{itemize}
                 As a consequence of these laws, the \haddockid{Functor} instance for \haddocktt{f} will satisfy\par
                 \begin{itemize}
                 \item
                 \begin{quote}
                 {\haddockverb\begin{verbatim}
                 fmap f x = pure f <*> x\end{verbatim}}
                 \end{quote}
                 
                 \end{itemize}
                 
\haddockpremethods{}\textbf{Methods}
\end{haddockdesc}
\begin{haddockdesc}
\item[\begin{tabular}{@{}l}
instance\ Applicative\ {\char 91}{\char 93}\\instance\ Applicative\ Maybe\\instance\ Applicative\ IO
\end{tabular}]
\end{haddockdesc}


\section{Alternative class
}

\begin{haddockdesc}
\item[\begin{tabular}{@{}l}
class\ Applicative\ f\ =>\ Alternative\ f\ where
\end{tabular}]\haddockbegindoc
A monoid on applicative functors.\par
If defined, \haddockid{some} and \haddockid{many} should be the least solutions
 of the equations:\par
\begin{itemize}
\item
\begin{quote}
{\haddockverb\begin{verbatim}
some v = fmap (:) v <*> many v\end{verbatim}}
\end{quote}

\item
\begin{quote}
{\haddockverb\begin{verbatim}
many v = some v <|> pure []\end{verbatim}}
\end{quote}

\end{itemize}

\haddockpremethods{}\textbf{Methods}
\end{haddockdesc}
\begin{haddockdesc}
\item[\begin{tabular}{@{}l}
instance\ Alternative\ {\char 91}{\char 93}\\instance\ Alternative\ Maybe
\end{tabular}]
\end{haddockdesc}

\section{Functions
}
\begin{haddockdesc}
\item[\begin{tabular}{@{}l}
(<**>)\ ::\ Applicative\ f\ =>\ f\ a\ ->\ f\ (a\ ->\ b)\ ->\ f\ b
\end{tabular}]\haddockbegindoc
A variant of \haddockid{<*>} with the arguments reversed.\par

\end{haddockdesc}
\begin{haddockdesc}
\item[\begin{tabular}{@{}l}
liftA\ ::\ Applicative\ f\ =>\ (a\ ->\ b)\ ->\ f\ a\ ->\ f\ b
\end{tabular}]\haddockbegindoc
Lift a function to actions.
 This function may be used as a value for \haddockid{fmap} in a \haddockid{Functor} instance.\par

\end{haddockdesc}
\begin{haddockdesc}
\item[\begin{tabular}{@{}l}
liftA3\ ::\ Applicative\ f\ =>\\\ \ \ \ \ \ \ \ \ \ (a\ ->\ b\ ->\ c\ ->\ d)\ ->\ f\ a\ ->\ f\ b\ ->\ f\ c\ ->\ f\ d
\end{tabular}]\haddockbegindoc
Lift a ternary function to actions.\par

\end{haddockdesc}
\begin{haddockdesc}
\item[\begin{tabular}{@{}l}
optional\ ::\ Alternative\ f\ =>\ f\ a\ ->\ f\ (Maybe\ a)
\end{tabular}]\haddockbegindoc
One or none.\par

\end{haddockdesc}
